%me=0 student solutions (ps file), me=1 - my solutions (sol file), me=2 - assignment (hw file)
\def\me{0}
\def\num{3}  %homework number
\def\due{Wednesday, February 18}  %due date
\def\course{CSCI-UA.0310-004/005 Basic Algorithms} %course name, changed only once
\def\name{Jason Yao}   %student changes (instructor keeps!)
%
\iffalse
INSTRUCTIONS: replace # by the homework number.
(if this is not ps#.tex, use the right file name)

  Clip out the ********* INSERT HERE ********* bits below and insert
appropriate TeX code.  Once you are done with your file, run

  ``latex ps#.tex''

from a UNIX prompt.  If your LaTeX code is clean, the latex will exit
back to a prompt.  To see intermediate results, type

  ``xdvi ps#.dvi'' (from UNIX prompt)
  ``yap ps#.dvi'' (if using MikTex in Windows)

after compilation. Once you are done, run

  ``dvips ps#.dvi''

which should print your file to the nearest printer.  There will be
residual files called ps#.log, ps#.aux, and ps#.dvi.  All these can be
deleted, but do not delete ps1.tex. To generate postscript file ps#.ps,
run

  ``dvips -o ps#.ps ps#.dvi''

I assume you know how to print .ps files (``lpr -Pprinter ps#.ps'')
\fi
%
\documentclass[11pt]{article}
\usepackage{amsfonts}
\usepackage{latexsym}
\usepackage{listings}
\setlength{\oddsidemargin}{.0in}
\setlength{\evensidemargin}{.0in}
\setlength{\textwidth}{6.5in}
\setlength{\topmargin}{-0.4in}
\setlength{\textheight}{8.5in}

\newcommand{\handout}[5]{
   \renewcommand{\thepage}{#1, Page \arabic{page}}
   \noindent
   \begin{center}
   \framebox{
      \vbox{
    \hbox to 5.78in { {\bf \course} \hfill #2 }
       \vspace{4mm}
       \hbox to 5.78in { {\Large \hfill #5  \hfill} }
       \vspace{2mm}
       \hbox to 5.78in { {\it #3 \hfill #4} }
      }
   }
   \end{center}
   \vspace*{4mm}
}

\newcounter{pppp}
\newcommand{\prob}{\arabic{pppp}}  %problem number
\newcommand{\increase}{\addtocounter{pppp}{1}}  %problem number

%first argument desription, second number of points
\newcommand{\newproblem}[2]{
\ifnum\me=0
\ifnum\prob>0 \newpage \fi
\increase
\setcounter{page}{1}
\handout{\name, Homework \num, Problem \arabic{pppp}}{\today}{Name: \name}{Due:
\due}{Solutions to Problem \prob\ of Homework \num\ (#2)}
\else
\increase
\section*{Problem \num-\prob~(#1) \hfill {#2}}
\fi
}

%\newcommand{\newproblem}[2]{\increase
%\section*{Problem \num-\prob~(#1) \hfill {#2}}
%}

\def\squarebox#1{\hbox to #1{\hfill\vbox to #1{\vfill}}}
\def\qed{\hspace*{\fill}
        \vbox{\hrule\hbox{\vrule\squarebox{.667em}\vrule}\hrule}}
\newenvironment{solution}{\begin{trivlist}\item[]{\bf Solution:}}
                      {\qed \end{trivlist}}
\newenvironment{solsketch}{\begin{trivlist}\item[]{\bf Solution Sketch:}}
                      {\qed \end{trivlist}}
\newenvironment{code}{\begin{tabbing}
12345\=12345\=12345\=12345\=12345\=12345\=12345\=12345\= \kill }
{\end{tabbing}}

\newcommand{\eqref}[1]{Equation~(\ref{eq:#1})}

\newcommand{\hint}[1]{({\bf Hint}: {#1})}
%Put more macros here, as needed.
\newcommand{\room}{\medskip\ni}
\newcommand{\brak}[1]{\langle #1 \rangle}
\newcommand{\bit}[1]{\{0,1\}^{#1}}
\newcommand{\zo}{\{0,1\}}
\newcommand{\C}{{\cal C}}

\newcommand{\nin}{\not\in}
\newcommand{\set}[1]{\{#1\}}
\renewcommand{\ni}{\noindent}
\renewcommand{\gets}{\leftarrow}
\renewcommand{\to}{\rightarrow}
\newcommand{\assign}{:=}

\newcommand{\AND}{\wedge}
\newcommand{\OR}{\vee}

\newcommand{\For}{\mbox{\bf For }}
\newcommand{\To}{\mbox{\bf to }}
\newcommand{\Do}{\mbox{\bf Do }}
\newcommand{\If}{\mbox{\bf If }}
\newcommand{\Then}{\mbox{\bf Then }}
\newcommand{\Else}{\mbox{\bf Else }}
\newcommand{\While}{\mbox{\bf While }}
\newcommand{\Repeat}{\mbox{\bf Repeat }}
\newcommand{\Until}{\mbox{\bf Until }}
\newcommand{\Return}{\mbox{\bf Return }}
\newcommand{\Swap}{\mbox{\bf Swap }}

\begin{document}

\ifnum\me=0
%\handout{PS\num}{\today}{Name: Jason Yao}{Due:
%\due}{Solutions to Problem Set \num}
%
%I collaborated with Katelyn Mulgrew on question 2 and question 4
\fi
\ifnum\me=1
\handout{PS\num}{\today}{Name: Yevgeniy Dodis}{Due: \due}{Solution
{\em Sketches} to Problem Set \num}
\fi
\ifnum\me=2
\handout{PS\num}{\today}{Lecturer: Yevgeniy Dodis}{Due: \due}{Problem
Set \num}
\fi

\lstset{ %
breaklines=true,        % sets automatic line breaking
breakatwhitespace=false,    % sets if automatic breaks should only happen at whitespace
showtabs=false,                 % show tabs within strings adding particular underscores
frame=single,           % adds a frame around the code
tabsize=2,          % sets default tabsize to 2 spaces
escapechar=?,
}

\newproblem{Stock Profit}{10 (+6) Points}

Sometimes, computing ``extra'' information can lead to more efficient
divide-and-conquer algorithms. As an example, we will improve on the
solution to the problem of maximizing the profit from investing in a
stock (page 68-74).

Suppose you are given an array $A$ of $n$ integers such that entry
$A[i]$ is the value of a particular stock at time interval $i$. The
goal is to find the time interval $(i,j)$ such that your profit is
maximized by buying at time $i$ and selling at time $j$. For example,
if the stock prices were monotone increasing, then $(1, n)$ would be
the interval with the maximal profit $(A[n]-A[1])$.  More formally,
the current formulation of the problem has the following input/output
specification:

\begin{description}
 \item{{\bf Input:}} Array $A$ of length $n$.
 \item{{\bf Output:}} Indices $i \le j$ maximizing $(A[j]-A[i])$.
\end{description}

\begin{itemize}
 \item[(a)] (6 Points) Suppose your change the input/output specification
of the
 stock problem to also compute the largest and the smallest stock
 prices:

\begin{description}
\item{{\bf Input:}} Array $A$ of length $n$.
\item{{\bf Output:}}
 Indices $i \le j$ maximizing $(A[j]-A[i])$, and indices $\alpha,
 \beta$ such that $A[\alpha]$ is a minimum of $A$ and $A[\beta]$ is a
 maximum of $A$.
\end{description}

Design a divide-and-conquer algorithm for this modified problem. Make
sure you try to design the most efficient ``conquer'' step, and argue
why it works. How long in your conquer step?
\\ \hint{When computing optimal $i$ and $j$, think whether the
``midpoint'' $n/2$ is less than $i$, greater than $j$ or in between
$i$ and $j$.}

\ifnum\me<2
\begin{solution}
\begin{lstlisting}
makeMoniesWrapper(A)
	?{\bf RETURN}? makeMonies(A, 1, A.length)
?{\bf END}? makeMoniesWrapper

makeMonies(A, i, j)
	// Base case: lines crossed, must buy before selling
	if (j < i)
		?{\bf RETURN}? 0
	endif
	mid = i + ?${\frac{j - i}{2}}$?
	maxLeft = makeMonies(A, i, mid)
	maxRight = makeMonies(A, mid + 1, j)
	?$\alpha$? = min(A[1:mid])
	?$\beta$? = max(A[mid + 1: j])
	
	?{\bf RETURN}? (maxLeft, maxRight, ?$\alpha$?, ?$\beta$?)
?{\bf END}? makeMonies
\end{lstlisting}
The conquer step takes $f(n) = \Theta(n)$, since it needs to find $\alpha$ and $\beta$, which should take 2n time, or $\Theta(n)$.

The conquer step works correctly because if you buy and then sell within the first half of array A, then maxLeft and maxRight will find the solution and return it.

If instead the min in in the first half, and the max is in the second half, then $\alpha$ and $\beta$ that was found will lead to the answer in the form $\beta - \alpha$

 \end{solution}
\fi

\item[(b)] (4 Points) Formally analyze the runtime of your algorithm and
compare it with the runtime of the solution for the Stock Profit
problem in the book.

\ifnum\me<2
\begin{solution}
Runtime of my algorithm:

$T(n) = 2T(\frac{n}{2}) + f(n)$

$T(n) = 2T(\frac{n}{2}) + \Theta(n)$

$T(n) = \Theta(n \log n)$

Runtime of the solution to the Stock Profit problem in the book

$T(n) = \Theta(n \log n)$, so we have matched the efficiency of the book's solution to the Stock Profit problem
 \end{solution}
\fi

\item[(c)] ({\bf Extra Credit}; 6 Points)
Design a direct, non-recursive algorithm for the Stock Profit
problem which runs in time $O(n)$. Write its pseudocode. Ideally,
you should have a single ``\For $i=1$ \To $n$'' loop, and inside the
loop you should maintain a few ``useful counters''. Formally argue
the correctness of your algorithm.\\
\hint{Scan the array left to right and maintain its running minimum and the
best solution found so far. Under which conditions would the best
current solution be improved when scanning the next array element?}

\ifnum\me<2
\newpage
\begin{solution}
\begin{lstlisting}
makeMonies(A)
bestMin, bestMax, ?$\alpha$? = 1
?$\beta$? = ?$\frac{A.length}{2}$?

?{\bf For}? i = 1 to n
	bestMin = min(A[bestMin], A[i])
	bestMax = max(A[bestMax], A[i])
	if (i ?$\leq \frac{n}{2}$? )	
		?$\alpha$? = min(A[?$\alpha$?], A[i])
		?$\beta$? = max(A[?$\beta$?], A[i + ?$\frac{A.length}{2}$?])
	endif
?{\bf endfor}?

?{\bf RETURN}? (bestMin, bestMax, ?$\alpha$?, ?$\beta$?)
?{\bf END}? makeMonies
\end{lstlisting}
\end{solution}
\fi
\end{itemize}


\newproblem{Local Minimum}{10 Points}

A {\em local minimum} of an array $A[1],\ldots,A[n]$ is an index $i\in
\{1,\ldots,n\}$ such that either (a) $i=1$ and $A[1]\le A[2]$; or (b)
$i=n$ and $A[n]\le A[n-1]$; or (c) $1<i<n$ and $A[i]\le A[i-1]$ and
$A[i]\le A[i+1]$. Note that every array has at least (and possibly
more than) one local minimum, since the ``global'' minumum of the
entire array is also a local minimum. Design an $O(\log n)$
divide-and-conquer algorithm to find some local minimum of a given
(unsorted) array $A$ of size $n$.\\
\hint{Think of binary search for inspiration.}



\ifnum\me<2
\begin{solution}

\begin{lstlisting}
findLocalMinWrapper(A)
	?{\bf RETURN}? findLocalMin(A, 1, A.length, A.length)
?{\bf END}? findLocalMinWrapper

findLocalMin(A, i, j, n)
	// Base case 1: Lines crossed, returns null value
	if (i == j)
		?{\bf RETURN}? 0
	endif
	mid = i + ?$\frac{j - i}{2}$?
	//Base case 2: Edge cases
	if (
		((mid == 1) && (A[mid] ?$\leq$? A[mid + 1])) ||
		((mid == n) && (A[mid] ?$\leq$? A[mid - 1])))
		?{\bf RETURN}? mid
	endif
	// Base case 3: Local min found
	if (
		(A[mid] ?$\leq$? A[mid - 1]) &&
		(A[mid] ?$\leq$? A[mid + 1])
		?{\bf RETURN}? mid
	endif
	// Recursive Steps
	p = findLocalMin(A, i, mid, n)
	if (p != 0)
		?{\bf RETURN}? p
	endif
	q = findLocalMin(A, mid + 1, j, n)
	if (q != 0)
		?{\bf RETURN}? q
	endif
	else
		?{\bf RETURN}? 0
	endelse
?{\bf END}? findLocalMin
\end{lstlisting}
 \end{solution}
\fi

\newproblem{Identity Element in an Array} {10 Points}

Let $A$ be an array with $n$ distinct integer elements in sorted order. Consider the following algorithm {\sc IdFind}$(A,j,k)$ that finds an $i\in \{j\ldots k\}$ such that $A[i] = i$, or returns {\bf FALSE} if no such element $i$ exists.

\begin{code}
 1 {\sc IdFind}$(A, j, k)$\\
 2 \> \If $j > k$ \Return {\bf FALSE} \\
 3 \> Set $i: = \ldots$\\
 4 \> \If $A[i] = \ldots$ \Return $\ldots$ \\
 5 \> \If $A[i] < \ldots$ \Return {\sc IdFind}$(A, \ldots, \ldots)$ \\
 6 \> \Return {\sc IdFind}$(A, \ldots, \ldots)$
\end{code}

\begin{itemize}
\item[(a)] (3 points)
Fill in the blanks (denoted $\ldots$) to complete the above algorithm.

\ifnum\me<2
\begin{solution}
\begin{code}
 1 {\sc IdFind}$(A, j, k)$\\
 2 \> \If $j > k$ \Return {\bf FALSE} \\
 3 \> Set $i: = j + \frac{k - j}{2}$\\
 4 \> \If $A[i] = i$ \Return {\bf TRUE} \\
 5 \> \If $A[i] < i$ \Return {\sc IdFind}$(A, i + 1, k)$ \\
 6 \> \Return {\sc IdFind}$(A, j, i)$
\end{code}
\end{solution}
\fi

\item[(b)] (5 points)
{\em Prove} correctness and analyze the running time of the algorithm.\\
(Notice the emphasis on {\em Prove}, you can't just say ``my algorithm works
because it works''.)

\ifnum\me<2
\begin{solution}

Input: Array $A$ of elements in sorted order

Output: boolean- TRUE or FALSE

Correctness: Proving I $\rightarrow$ O

	\hspace{.05\textwidth} Part 1: Output check
	
	\hspace{.1\textwidth}Case proofs:

		\hspace{.1\textwidth}If there is no element found, returns false
		
		\hspace{.1\textwidth}If there an element is found, returns true
		
		\hspace{.1\textwidth}If the element is lower, call the method again with the lower half of the sub-array, returns a boolean
		
		\hspace{.1\textwidth}If the element is higher, call the method again with the upper half of the sub array, returns a boolean
	
	\hspace{.05\textwidth}Part 2: Algorithm Terminates
	
	\hspace{.05\textwidth}As all four potential cases returns a value, the algorithm thus terminates correctly, as it returns a boolean in each case
	
	Running Time:
	
	$T(n) = 2T(\frac{n}{2}) + 1$
	
	$T(n) = \Theta(\log n)$
 \end{solution}
\fi

\item[(c)] (2 points)
Does the algorithm work if the elements of $A$ are not distinct? Why or why not?
\ifnum\me<2
\begin{solution}
The algorithm does not work if there is repeated elements in A, since the duplicated element in A would cause a shift towards the right in all subsequent elements, such that eventually it would return false.

i.e. the array A = {1,2,2,5,6,7}, in which the algorithm would return false, since it would finish searching the right side, and then return false, even though i = 2 would be the normally correct answer
 \end{solution}
\fi

\end{itemize}

\newproblem{QuickSort vs Insertion Sort}{10 Points}

We say that an array $A$ is {\em $c$-nice}, where $c$ is a constant
(think $100$), if for all $1\le i,j \le n$, such that $j-i\ge c$, we
have that $A[i]\le A[j]$. For example, $1$-nice array is already
sorted. In this problem we will sort such $c$-nice $A$ using {\sc
Insertion Sort} and {\sc QuickSort}, and compare the results.

\begin{itemize}
 \item[(a)] (4 Points) In asymptotic notation (remember, $c$ is a
 constant) what is the worst-case running time of {\sc Insertion Sort}
 on a $c$-nice array?  Be sure to justify your answer.

\ifnum\me<2
\begin{solution}

As C is a relatively small number in an n-element array A, that means that there are a greater number of indicies $i$ and $j$ that are already in the correct relative position, indicating that Insertion Sort will not have to do many swaps, since it's mostly sorted.

Thus the worst case running time of insertion sort in this instance is in linear time, or $T_n = \Theta(n)$
 \end{solution}
\fi

\item[(b)] (5 Points) In asymptotic notation (remember, $c$ is a constant)
 what is the  worst-case running time of {\sc QuickSort} on a $c$-nice array?
Be sure to justify your answer.

\ifnum\me<2
\begin{solution}

As C is a relatively small number in an n-element array A, that means that there are a greater number of indicies $i$ and $j$ that are already in the correct relative position, indicating that Quicksort will have to go through a basically pre-sorted array, causing it's worst case to occur, and for it to sort it in $T_n = \Theta(n^2)$
 \end{solution}
\fi

\item[(c)] (1 Points) Which algorithm would you prefer?

\ifnum\me<2
\begin{solution}

Due to the fact that the array A is near pre-sorted, and is thus Insertion Sort's best case scenario, and Quicksort's worst case scenario, I would prefer to use Insertion Sort in this instance.
\end{solution}
\fi
\end{itemize}


\end{document}


