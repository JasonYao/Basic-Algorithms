%me=0 student solutions, me=1 - my solutions, me=2 - assignment
\def\me{0}
\def\num{2}  %homework number
\def\due{Wednesday, February 11}  %due date
\def\course{CSCI-UA.0310-004/005 Basic Algorithms} %course name
\def\name{Jason Yao}
%
\iffalse
INSTRUCTIONS: replace # by the homework number.
(if this is not ps#.tex, use the right file name)

  Clip out the ********* INSERT HERE ********* bits below and insert
appropriate TeX code.  Once you are done with your file, run

  ``latex ps#.tex''

from a UNIX prompt.  If your LaTeX code is clean, the latex will exit
back to a prompt.  To see intermediate results, type

  ``xdvi ps#.dvi'' (from UNIX prompt)
  ``yap ps#.dvi'' (if using MikTex in Windows)

after compilation. Once you are done, run

  ``dvips ps#.dvi''

which should print your file to the nearest printer.  There will be
residual files called ps#.log, ps#.aux, and ps#.dvi.  All these can be
deleted, but do not delete ps1.tex. To generate postscript file ps#.ps,
run

  ``dvips -o ps#.ps ps#.dvi''

I assume you know how to print .ps files (``lpr -Pprinter ps#.ps'')
\fi
%
\documentclass[11pt]{article}
\usepackage{amsfonts}
\usepackage{listings}
\usepackage{latexsym}
\setlength{\oddsidemargin}{.0in}
\setlength{\evensidemargin}{.0in}
\setlength{\textwidth}{6.5in}
\setlength{\topmargin}{-0.4in}
\setlength{\textheight}{8.5in}

\newcommand{\handout}[5]{
   \renewcommand{\thepage}{#1, Page \arabic{page}}
   \noindent
   \begin{center}
   \framebox{
      \vbox{
    \hbox to 5.78in { {\bf \course} \hfill #2 }
       \vspace{4mm}
       \hbox to 5.78in { {\Large \hfill #5  \hfill} }
       \vspace{2mm}
       \hbox to 5.78in { {\it #3 \hfill #4} }
      }
   }
   \end{center}
   \vspace*{4mm}
}

\newcounter{pppp}
\newcommand{\prob}{\arabic{pppp}}  %problem number
\newcommand{\increase}{\addtocounter{pppp}{1}}  %problem number

%first argument desription, second number of points
\newcommand{\newproblem}[2]{
\ifnum\me=0
\ifnum\prob>0 \newpage \fi
\increase
\setcounter{page}{1}
\handout{\name, Homework \num, Problem \arabic{pppp}}{\today}{Name: \name}{Due:
\due}{Solutions to Problem \prob\ of Homework \num\ (#2)}
\else
\increase
\section*{Problem \num-\prob~(#1) \hfill {#2}}
\fi
}

%\newcommand{\newproblem}[2]{\increase
%\section*{Problem \num-\prob~(#1) \hfill {#2}}
%}

\def\squarebox#1{\hbox to #1{\hfill\vbox to #1{\vfill}}}
\def\qed{\hspace*{\fill}
        \vbox{\hrule\hbox{\vrule\squarebox{.667em}\vrule}\hrule}}
\newenvironment{solution}{\begin{trivlist}\item[]{\bf Solution:}}
                      {\qed \end{trivlist}}
\newenvironment{solsketch}{\begin{trivlist}\item[]{\bf Solution Sketch:}}
                      {\qed \end{trivlist}}
\newenvironment{code}{\begin{tabbing}
12345\=12345\=12345\=12345\=12345\=12345\=12345\=12345\= \kill }
{\end{tabbing}}

\newcommand{\eqref}[1]{Equation~(\ref{eq:#1})}

\newcommand{\hint}[1]{({\bf Hint}: {#1})}
%Put more macros here, as needed.
\newcommand{\room}{\medskip\ni}
\newcommand{\brak}[1]{\langle #1 \rangle}
\newcommand{\bit}[1]{\{0,1\}^{#1}}
\newcommand{\zo}{\{0,1\}}
\newcommand{\C}{{\cal C}}

\newcommand{\nin}{\not\in}
\newcommand{\set}[1]{\{#1\}}
\renewcommand{\ni}{\noindent}
\renewcommand{\gets}{\leftarrow}
\renewcommand{\to}{\rightarrow}
\newcommand{\assign}{:=}

\newcommand{\AND}{\wedge}
\newcommand{\OR}{\vee}

\newcommand{\For}{\mbox{\bf For }}
\newcommand{\To}{\mbox{\bf to }}
\newcommand{\Do}{\mbox{\bf Do }}
\newcommand{\If}{\mbox{\bf If }}
\newcommand{\Then}{\mbox{\bf Then }}
\newcommand{\Else}{\mbox{\bf Else }}
\newcommand{\While}{\mbox{\bf While }}
\newcommand{\Repeat}{\mbox{\bf Repeat }}
\newcommand{\Until}{\mbox{\bf Until }}
\newcommand{\Return}{\mbox{\bf Return }}
\newcommand{\Swap}{\mbox{\bf Swap }}

\begin{document}

\ifnum\me=0
%\handout{PS\num}{\today}{Name: Jason Yao}{Due:
%\due}{Solutions to Problem Set \num}
%
%I collaborated with *********** INSERT COLLABORATORS HERE (INDICATING
%SPECIFIC PROBLEMS) *************.
\fi
\ifnum\me=1
\handout{PS\num}{\today}{Name: Jason Yao}{Due: \due}{Solution
{\em Sketches} to Problem Set \num}
\fi
\ifnum\me=2
\handout{PS\num}{\today}{Lecturer: Yevgeniy Dodis}{Due: \due}{Problem
Set \num}
\fi

\newproblem{Counting Inversions, Part 1}{10  points}

Let $A[1, \ldots, n]$ be an array of $n$ distinct numbers.
If $i < j$ and $A[i] > A[j]$, then the pair $(i,j)$ is called
an {\bf{inversion}} of $A$.

\begin{itemize}

\item[(a)] (2 points)
List the five inversions of the array $\langle 2,3,8,6,1 \rangle$.

\ifnum\me<2
\begin{solution}
(3, 4), (1, 5),(2, 5),(3, 5), (4, 5)
\end{solution}
\fi

\item[(b)] (3 points) Which arrays with elements from the set
  $\{1,2,\ldots, n\}$ have the smallest and the largest number of inversions and why?
  State the expressions {\em exactly} in terms of $n$.

\ifnum\me<2
\begin{solution}

Smallest number of inversions:

	Array is sorted in ascending order, which by definition has no inversions. $\Theta = (n)$

Largest number of inversions:

	Array is sorted in descending order, which by definition means that there are n + (n - 1) + ... + 1 inversions, and thus
	$\Theta = (n^{2})$
\end{solution}
\fi

\item[(c)] (5 points) What is the relationship between the running time of
  {\sc{Insertion\_sort}} and the number of inversions $I$ in the input
  array? {\em Justify your answer}.

\ifnum\me<2
\begin{solution}
Insertion sort method relies upon the idea of swapping two elements in an array if there is an inversion. Thus, the more inversions there are in the array, the longer the running time. Conversely, fewer inversions in the array results in a shorter running time
\end{solution}
\fi

\end{itemize}

\newproblem{Counting Inversions, Part 2}{16 points}

Let $A[1 \ldots n]$ be an array of pairwise different numbers. We call
pair of indices $1 \leq i < j \leq n$ an
\emph{inversion} of $A$ if $A[i] > A[j]$. The goal of this problem is
to develop a divide-and-conquer based algorithm running in time
$\Theta(n \log n)$ for computing the number of inversions in $A$.
\begin{itemize}
 \item[(a)] (8 points) Suppose you are given a pair of {\em sorted} integer
 arrays $A$ and $B$ of length $n/2$ each. Let $C$ an $n$-element array
 consisting of the concatenation of $A$ followed by $B$. Give an
 algorithm (in pseudocode) for counting the number of inversions in
 $C$ and analyze its runtime. Make sure you also argue (in English)
 why your algorithm is correct.

\lstset{ %
breaklines=true,        % sets automatic line breaking
breakatwhitespace=false,    % sets if automatic breaks should only happen at whitespace
showtabs=false,                 % show tabs within strings adding particular underscores
frame=single,           % adds a frame around the code
tabsize=2,          % sets default tabsize to 2 spaces
}

\ifnum\me<2
\begin{solution}
\begin{lstlisting}
inversionCount(A[], B[], C[])
i = 0
j = A.length
counter = 0

while (i < A.length) && (j < C.length)
	if C[i] > C[j]
		++counter
		++j
	endif
	else // No more inversions occurring, can skip the rest
		++i
		j = A.length // Resets for new round of comparisons
	endelse
endwhile

return counter
\end{lstlisting}

Runtime:

The algorithm's worst-case runtime is $T(n) = \Theta (n^2)$, since the while loop acts the same way as an outer for loop in a nested version of the code, in this case with resetting j.

Algorithm Correctness:

The algorithm depends upon the property of C[] in that it contains two pre-sorted arrays, meaning that it is possible to skip checking after an inversion check fails, since every other check after that would be unnecessary.

Thus, all the algorithm is doing is running a counter, and running until the first inversion check fails, then resetting, and doing it again with a different i value.
\end{solution}
\fi

 \item[(b)] (8 points) Give an algorithm (in pseudocode) for counting the number
of inversions in an $n$ element array $A$ that runs in time $\Theta(n
\log n)$. Make sure you formally prove that your algorithm runs in
time $\Theta(n \log n)$ (e.g., write the recurrence and solve it.) \\
\hint{Combine Merge Sort with part (a).}

\ifnum\me<2
\begin{solution}
\begin{lstlisting}

mergedInversionCountWrapper(A[])
	mergedInversionCountRec(A, 0, A.length - 1)
end mergedInversionCounterWrapper

mergedInversionCountRec(A[], start, end)
	// Base case
	if start == end
		return
	endif
	
	// Recursive case
	mid = (end - start)/2 + start
	mergedInversionCountRec(A, start, mid)
	mergedInversionCountRec(A, mid + 1, end)
	mergeCount(A, start, mid, end)
end mergedInversionCountRec

mergeCount (A[], start, mid, end)
	i = start
	j = end
	counter = 0

while (i < mid) && (j < end)
	if C[i] > C[j]
		++counter
		++j
	endif
	else // No more inversions occurring, can skip the rest
		++i
		j = mid // Resets for new round of comparisons
	endelse
endwhile

return counter
end mergeCount
\end{lstlisting}

\end{solution}
\fi
\end{itemize}

\newproblem{The Same Outcome in Different Ways}{15 Points}

Consider the following recurrence  $T(n) = 4 T(n/2) + n^2 \log n$,
$T(1)=1$.

\begin{itemize}

\item[(a)] (2 points)
Can the master's theorem, as stated in the book, be applied to solve
this recurrence? If yes, apply it. If not, formally explain the reason
why.

\ifnum\me<2
\begin{solution}

$a = 4, b = 2, f(n) = n^{2} \log n$

$n^{\log_2 4}$ ? $n^2 \log n$

$n^{2}$ ? $n^2 \log n$

As this case falls between cases 2 and 3 ($f(n)$ is larger than $n^{log_b a}$, but not polynomially larger), the master theorem  cannot be used to solve this recurrence.
\end{solution}
\fi

\item[(b)] (4 points)
Solve the above recurrence using the recursion tree method.

\ifnum\me<2
\begin{solution}

Assumptions: n is an exact power of 2

$T(n) = cn^2 \log n + 2 cn^2 \log n + 4cn^2 \log n + ...$
$+ 2^n cn^2 \log n$

$T(n) = \sum\limits_{i=1}^n 2^i cn^2 \log n + \Theta(n^2)$

$T(n) = O(2^n)$
\end{solution}
\fi

\item[(c)] (4 points)
Formally verify that your answer from part (b) is correct using
induction.

\ifnum\me<2
\begin{solution}

$T(n) \leq c * 2^n$

Base case:
n = 1

T(1) = 1 ? $\leq c * 2^{(1)}$

Okay as long as $c \geq 1$

Inductive Step:

Assume true for 1, 2, ..., n

$T(n) = 4T(n/2) + n^2 \log n \leq 4(c * 2^n) + (c * 2^n)^{2} \log c * 2^n$

$n^2 ? \leq 2^n$
\end{solution}
\fi

\item[(d)] (5 points)
Solve the above recurrence exactly using domain-range substitution.

\ifnum\me<2
\begin{solution}

Domain Substitution:

$T(n) = a T(\frac{n}{b}) + f(n)$

$T(n) = 4 T(n/2) + n^2 \log n$,
$T(1)=1$

Let n = $2^k$

$T(b^k) = 4T(\frac{2^{k}}{2}) + (2^{2k}) k \log 2$

$S(k) = 4S(2^{k-1}) + (2^{2k}) k, S(0) = T(1) = 1 = f(1)$

Applying Range Substitution:

$R(n) = \frac{S(k)}{2^k} = \frac{4S(2^{k-1}) + (2^{2k}) k}{2^k} = 2^{2-k}S(2^{k-1}) + (2^{k}) k$

$S(k) = (2^{2k + 1}) k \log 2$

$k = \log n$

$T(n) = S(\log n) = \Theta(2^n)$
\end{solution}
\fi

\end{itemize}

\newproblem{Tricky Recurrence}{5 points}

Solve {\em precisely} the recurrence $T(n)  = T(\sqrt{n}) + \log n$,
with $T(2)=2$ (assume $n$ is such that $\sqrt{\sqrt{\ldots \sqrt{n}}}$
is always an integer).
\hint{Substitute $n=2^k$... Then substitute again.}

\ifnum\me<2
\begin{solution}

$T(n) = T(\sqrt{n}) + \log n, T(2) = 2$

Substitute $n = 2^k (k = \log_2 n)$

$T(n) = T(2^k) = T(\sqrt{2^k}) + \log_2 2^k$

$T(2^k) = T(\sqrt{2^k}) + k$

$S(k) = S(k/2) + k, S(1) = 1$

Substitute $k = 2^m (m = \log_2 k)$

$S(k) = S(2^m) = S(\frac{2^m}{2}) + 2^m$

$S(2^m) = S(2^{m - 1}) + 2^m$

$R(m) = R(m - 1) + 2^m, R(0) = 1$

$R(m) = 2^{m - 1} - 1$

Back-solving, $m = \log_2 k$

$R(\log_2 k) = 2^{\log_2 k + 1} - 1$

$R(\log_2 k) = 2k - 1$

Back-solving, $k = \log_2 n$

$R(\log_2 \log_2 n) = 2(\log_2 n) - 1$

$T(n) = R(\log_2 \log_2 n) = 2 \log_2 n - 1$

$T(n) = 2 \log_2 n - 1$
\end{solution}
\fi

\end{document}


