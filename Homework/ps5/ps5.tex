%me=0 student solutions (ps file), me=1 - my solutions (sol file), me=2 - assignment (hw file)
\def\me{0}
\def\num{5}  %homework number
\def\due{Wednesday, March 4}  %due date
\def\course{CSCI-UA.0310-004/005 Basic Algorithms} %course name, changed only once
\def\name{Jason Yao}   %student changes (instructor keeps!)
%
\iffalse
INSTRUCTIONS: replace # by the homework number.
(if this is not ps#.tex, use the right file name)

  Clip out the ********* INSERT HERE ********* bits below and insert
appropriate TeX code.  Once you are done with your file, run

  ``latex ps#.tex''

from a UNIX prompt.  If your LaTeX code is clean, the latex will exit
back to a prompt.  To see intermediate results, type

  ``xdvi ps#.dvi'' (from UNIX prompt)
  ``yap ps#.dvi'' (if using MikTex in Windows)

after compilation. Once you are done, run

  ``dvips ps#.dvi''

which should print your file to the nearest printer.  There will be
residual files called ps#.log, ps#.aux, and ps#.dvi.  All these can be
deleted, but do not delete ps1.tex. To generate postscript file ps#.ps,
run

  ``dvips -o ps#.ps ps#.dvi''

I assume you know how to print .ps files (``lpr -Pprinter ps#.ps'')
\fi
%
\documentclass[11pt]{article}
\usepackage{amsfonts}
\usepackage{latexsym}
\usepackage{listings}
\setlength{\oddsidemargin}{.0in}
\setlength{\evensidemargin}{.0in}
\setlength{\textwidth}{6.5in}
\setlength{\topmargin}{-0.4in}
\setlength{\textheight}{8.5in}

\newcommand{\handout}[5]{
   \renewcommand{\thepage}{#1, Page \arabic{page}}
   \noindent
   \begin{center}
   \framebox{
      \vbox{
    \hbox to 5.78in { {\bf \course} \hfill #2 }
       \vspace{4mm}
       \hbox to 5.78in { {\Large \hfill #5  \hfill} }
       \vspace{2mm}
       \hbox to 5.78in { {\it #3 \hfill #4} }
      }
   }
   \end{center}
   \vspace*{4mm}
}

\newcounter{pppp}
\newcommand{\prob}{\arabic{pppp}}  %problem number
\newcommand{\increase}{\addtocounter{pppp}{1}}  %problem number

%first argument desription, second number of points
\newcommand{\newproblem}[2]{
\ifnum\me=0
\ifnum\prob>0 \newpage \fi
\increase
\setcounter{page}{1}
\handout{\name, Homework \num, Problem \arabic{pppp}}{\today}{Name: \name}{Due:
\due}{Solutions to Problem \prob\ of Homework \num\ (#2)}
\else
\increase
\section*{Problem \num-\prob~(#1) \hfill {#2}}
\fi
}

%\newcommand{\newproblem}[2]{\increase
%\section*{Problem \num-\prob~(#1) \hfill {#2}}
%}

\def\squarebox#1{\hbox to #1{\hfill\vbox to #1{\vfill}}}
\def\qed{\hspace*{\fill}
        \vbox{\hrule\hbox{\vrule\squarebox{.667em}\vrule}\hrule}}
\newenvironment{solution}{\begin{trivlist}\item[]{\bf Solution:}}
                      {\qed \end{trivlist}}
\newenvironment{solsketch}{\begin{trivlist}\item[]{\bf Solution Sketch:}}
                      {\qed \end{trivlist}}
\newenvironment{code}{\begin{tabbing}
12345\=12345\=12345\=12345\=12345\=12345\=12345\=12345\= \kill }
{\end{tabbing}}

\newcommand{\eqref}[1]{Equation~(\ref{eq:#1})}

\newcommand{\hint}[1]{({\bf Hint}: {#1})}
%Put more macros here, as needed.
\newcommand{\room}{\medskip\ni}
\newcommand{\brak}[1]{\langle #1 \rangle}
\newcommand{\bit}[1]{\{0,1\}^{#1}}
\newcommand{\zo}{\{0,1\}}
\newcommand{\C}{{\cal C}}

\newcommand{\nin}{\not\in}
\newcommand{\set}[1]{\{#1\}}
\renewcommand{\ni}{\noindent}
\renewcommand{\gets}{\leftarrow}
\renewcommand{\to}{\rightarrow}
\newcommand{\assign}{:=}

\newcommand{\AND}{\wedge}
\newcommand{\OR}{\vee}

\newcommand{\For}{\mbox{\bf For }}
\newcommand{\To}{\mbox{\bf to }}
\newcommand{\Do}{\mbox{\bf Do }}
\newcommand{\If}{\mbox{\bf If }}
\newcommand{\Then}{\mbox{\bf Then }}
\newcommand{\Else}{\mbox{\bf Else }}
\newcommand{\While}{\mbox{\bf While }}
\newcommand{\Repeat}{\mbox{\bf Repeat }}
\newcommand{\Until}{\mbox{\bf Until }}
\newcommand{\Return}{\mbox{\bf Return }}
\newcommand{\Swap}{\mbox{\bf Swap }}

\begin{document}

\ifnum\me=0
%\handout{PS\num}{\today}{Name: Jason Yao}{Due:
%\due}{Solutions to Problem Set \num}
%
%I collaborated with *********** INSERT COLLABORATORS HERE (INDICATING
%SPECIFIC PROBLEMS) *************.
\fi
\ifnum\me=1
\handout{PS\num}{\today}{Name: Jason Yao}{Due: \due}{Solution
{\em Sketches} to Problem Set \num}
\fi
\ifnum\me=2
\handout{PS\num}{\today}{Lecturer: Yevgeniy Dodis}{Due: \due}{Problem
Set \num}
\fi

\lstset{ %
breaklines=true,        % sets automatic line breaking
breakatwhitespace=false,    % sets if automatic breaks should only happen at whitespace
showtabs=false,                 % show tabs within strings adding particular underscores
frame=single,           % adds a frame around the code
tabsize=2,          % sets default tabsize to 2 spaces
escapechar=?,
}

\newproblem{Sorting in $n \log\!\log n$ Time}{6
points}

\begin{itemize}

\item[(a)] Assume you are given an array of $n$ integers with many
duplications, so that you know that there are at most $\log n$
distinct elements in the array. Show how to sort this array in time
$O(n \log\!\log n)$.

\ifnum\me<2

\begin{solution}

Let $k = \log n$

Let each node contain:

- Data value

- Number of times it occurs

By implementing HEAP-SORT, it is possible to get $2*n*log k$


\begin{lstlisting}
fastDuplicateSort(A[])
n = A.length
for i = 1 to n
	INSERT(A[i])
endfor

for j = 1 to n
	EXTRACT-MAX()
endfor
?{\bf END}? fastDuplicateSort
\end{lstlisting}

Runtime Analysis:

As each for loop is running in $T(n) = \log k + n \log k$, that gives a total runtime of $T(n) = 2(\log k + n \log k)$, or $T(n) = \Theta(\log k + n \log k)$

$T(n) = \Theta(n \log k)$

$T(n) = \Theta(n \log \log n)$
\end{solution}
\fi

\item[(b)] Assume you are given an array of $n$ integers in the range
$\set{1,\ldots, (\log n)^{\log n}}$. Show how to sort this array in time
$O(n \log\!\log n)$.

\ifnum\me<2
\begin{solution}   ***************** INSERT YOUR SOLUTION HERE ***************   \end{solution}
\fi

\end{itemize}

\newproblem{Choosing the Right Tool}{9 points}

For each example choose one of the following sorting algorithms and
carefully justify your choice: {\sc HeapSort}, {\sc RadixSort}, {\sc
CountingSort}. Give the expected runtime for your choice as precisely as
possible. If you choose Radix Sort then give a concrete choice for
the basis (i.e. the value of ``$r$'' in the book) and justify it.
\hint{We assume that the array itself is stored in memory, so before
choosing the fastest algorithm, make sure you have the space to run it!}

\begin{itemize}
 \item[(a)] Sort the length $2^{16}$ array $A$ of $128$-bit integers on a
device with $100$MB of RAM.

\ifnum\me<2
\begin{solution}

16 Bytes per integer, Memory already in use: $(2^{16})*16 = 1.048576 MB$

98.951424MB remains, compared to 1.048576 MB that was used to load the integers to memory

Maximum represented value: $2^{128} - 1$, which is too large to implement Radix or Counting Sort, since the k value (maximum value) is too large, and would not be efficient.

Thus, HEAPSORT would be the recommended sorting algorithm, which is an in-place sorting algorithm, with a running time of $T(n) = n \log n$.
\end{solution}
\fi


 \item[(b)] Sort the length $2^{24}$ array $A$ of $256$-bit integers on a
device with $600$MB of RAM.

\ifnum\me<2
\begin{solution}

32 Bytes per integer, Memory already in use: $(2^{24})*32 = 536.870912 MB$

63.129088MB is available for the sorting, meaning that an in-place, or near in place sort would be required due to memory constraints.

Maximum represented value: $2^{256} - 1$, which would be too big to implement Radix Sort or Counting Sort, since the k value (maximum value) is too large, and would not be efficient.

Thus, HEAPSORT would be the recommended sorting algorithm, which is an in-place sorting algorithm, with a running time of $T(n) = n \log n$.
\end{solution}
\fi


 \item[(c)] Sort the length $2^{16}$ array $A$ of $16$-bit integers on a
device with $1$GB of RAM.

\ifnum\me<2
\begin{solution}

2 Bytes per integer, Memory already in use: $(2^{16})*2 = 0.131072 MB$

Over 999.869MB remains available for sorting, a huge amount of memory, especially considering the small amount of memory the integers take up.

Maximum represented value: $2^{16} - 1 = 65,535$, which is small enough that we can implement Radix Sort and Counting Sort.

In this case, as Counting Sort would be  which would run in $T(n) = O(k + n)$, in which k is the number of digits each integer has.

In this case, $T(n) = O(16 + n)$, $T(n) = O(n)$
\end{solution}
\fi
\end{itemize}

\newproblem{Lower Bound for Min-Element}{16 points}

Recall that there exists a trivial algorithm for searching for a minimal element of an array of size $n$
 that needs exactly $(n-1)$ comparisons. The purpose of this problem is to prove that this is optimal solution.

You can assume that elements of the input array are distinct. As usual, you can represent any comparison-based algorithm for Min-Element problem as a binary decision tree, whose internal nodes are labeled by $(i:j)$ (meaning ``compare $A[i]$ and $A[j]$''), and a left child is chosen if and only iff $A[i] < A[j]$ (recall, we assume distinct elements). And the leaves are labeled by the index $i$ meaning that the smallest element in the arrange is $A[i]$.

\begin{itemize}
\item[(a)] (4 points warm-up)
Show that the naive ``counting leaves'' application of decisional tree method (ala sorting lower bound) will only give a very suboptimal lower bound $\Omega(\log n)$ for the Min-Element problem.

\ifnum\me<2
\begin{solution}

As the height of a binary decision tree is given as $\log n$, and getting to the leaves (finding the lower bound) would take that amount of node traversals, the counting leaves of decisonal trees would end up giving $\Omega(\log n)$
\end{solution}
\fi

\item[(b)] (4 points) Here we will try to enrich the decisional tree by adding some additional info $S(v)$ to every vertex $v$. Precisely, $S(v)$ is the set of all indices $i$ which are ``consistent'' with all the comparisons made from the $root$ to $v$ (excluding $v$ itself), where ``consistent'' means that there exists an array $A$ whose minimum is $A[i]$ and the node $v$ is reached on input $i$. For example, if $v$ is any (reachable) leaf labeled by $i$, then $S(v) = \{i\}$ (as otherwise the algorithm would not be correct). Assuming the root node $root$ is labeled $(i:j)$, describe  $S(\textnormal{root})$, $S(\textnormal{root.left})$ and $S(\textnormal{root.right})$.

\hint{What element is impossible to be minimal if you know that $A[i] < A[j]$?}

\ifnum\me<2
\begin{solution}

The element that is to the right of the root
S(root) would contain the set of all possible combinations to find the minimal value

S(Root.LEFT) contains the set of (n-1) comparisons, as there is now one less element that needs to be compared.

S(Root.RIGHT) contains the set of just 1 element, which was the element j such that it is greater than i.
\end{solution}
\fi

\item[(c)] (4 points) Generalize part (b): show that for every vertex $v$ we have:
$S(v.\textnormal{left}) = S(v) \setminus \{ \ldots \}$ and
$S(v.\textnormal{right}) = S(v) \setminus \{ \ldots \}$ (you need to fill dots on your own).

\ifnum\me<2
\begin{solution}

For every vertex $v$, by definition, has another two vertices v.left and v.right, such that S(v.left) = \{the previous i\}, and S(v.right) = \{the previous j\}
\end{solution}
\fi

\item[(d)] (4 points) Use (c) to prove that in any valid decision tree for Min-Element, {\em any} leaf (which is stronger than {\em some} leaf)  must have depth at least $(n-1)$.
\hint{Think about $|S(v)|$ as $v$ goes from root to this leaf.}

\ifnum\me<2
\begin{solution}

As you have to do n - 1 comparisons due to having to compare against all values in S, the depth of all leaves in the decision tree must be at least n - 1, as $|S(v)|$ decreases from all possible combinations to 1 by the time it reaches a leaf.


\end{solution}
\fi

\end{itemize}

\newproblem{Roller coaster}{10 points}
We will call an array $B[1\ldots n]$ \emph{a roller coaster} if $B[1] < B[2]$, $B[2] > B[3]$, $B[3] < B[4]$ and so on. More formally: $B[2i] > B[2i +1 ]$ and $B[2i-1] < B[2i]$ for all $i$.

Give a linear algorithm that any given array $A$ with $n$ distinct elements transforms into a roller coaster array $B$. Namely, $B$ must contain exactly the same $n$ distinct elements as $A$, but must also be a roller coaster.
\hint{A median may be useful here.}

\ifnum\me<2
\begin{solution}

\begin{lstlisting}
makeRollerCoaster(A[])

n = A.length
// Sorts the array A
A = mergeSort(A)

// Takes an element from the end and the front, and builds a new array that alternates mins and maxes

for i = 1 to n/2, i += 2
	B[i] = A[i]
	B[i + 1] = A[n - i]
endfor

?{\bf END}? makeRollerCoaster
\end{lstlisting}

\end{solution}
\fi

\end{document}
