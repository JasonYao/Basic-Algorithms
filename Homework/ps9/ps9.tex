%me=0 student solutions (ps file), me=1 - my solutions (sol file), me=2 - assignment (hw file)
\def\me{0}
\def\num{9}  %homework number
\def\due{Wednesday, April 15}  %due date
\def\course{CSCI-UA.0310-004/005 Basic Algorithms} %course name, changed only once
\def\name{Jason Yao}   %student changes (instructor keeps!)
%
\iffalse
INSTRUCTIONS: replace # by the homework number.
(if this is not ps#.tex, use the right file name)

  Clip out the ********* INSERT HERE ********* bits below and insert
appropriate TeX code.  Once you are done with your file, run

  ``latex ps#.tex''

from a UNIX prompt.  If your LaTeX code is clean, the latex will exit
back to a prompt.  To see intermediate results, type

  ``xdvi ps#.dvi'' (from UNIX prompt)
  ``yap ps#.dvi'' (if using MikTex in Windows)

after compilation. Once you are done, run

  ``dvips ps#.dvi''

which should print your file to the nearest printer.  There will be
residual files called ps#.log, ps#.aux, and ps#.dvi.  All these can be
deleted, but do not delete ps1.tex. To generate postscript file ps#.ps,
run

  ``dvips -o ps#.ps ps#.dvi''

I assume you know how to print .ps files (``lpr -Pprinter ps#.ps'')
\fi
%
\documentclass[11pt]{article}
\usepackage{amsfonts}
\usepackage{listings}
\usepackage{latexsym}

%\usepackage{tikz}

\setlength{\oddsidemargin}{.0in}
\setlength{\evensidemargin}{.0in}
\setlength{\textwidth}{6.5in}
\setlength{\topmargin}{-0.4in}
\setlength{\textheight}{8.5in}

\newcommand{\handout}[5]{
   \renewcommand{\thepage}{#1, Page \arabic{page}}
   \noindent
   \begin{center}
   \framebox{
      \vbox{
    \hbox to 5.78in { {\bf \course} \hfill #2 }
       \vspace{4mm}
       \hbox to 5.78in { {\Large \hfill #5  \hfill} }
       \vspace{2mm}
       \hbox to 5.78in { {\it #3 \hfill #4} }
      }
   }
   \end{center}
   \vspace*{4mm}
}

\newcounter{pppp}
\newcommand{\prob}{\arabic{pppp}}  %problem number
\newcommand{\increase}{\addtocounter{pppp}{1}}  %problem number

%first argument desription, second number of points
\newcommand{\newproblem}[2]{
\ifnum\me=0
\ifnum\prob>0 \newpage \fi
\increase
\setcounter{page}{1}
\handout{\name, Homework \num, Problem \arabic{pppp}}{\today}{Name: \name}{Due:
\due}{Solutions to Problem \prob\ of Homework \num\ (#2)}
\else
\increase
\section*{Problem \num-\prob~(#1) \hfill {#2}}
\fi
}

%\newcommand{\newproblem}[2]{\increase
%\section*{Problem \num-\prob~(#1) \hfill {#2}}
%}

\def\squarebox#1{\hbox to #1{\hfill\vbox to #1{\vfill}}}
\def\qed{\hspace*{\fill}
        \vbox{\hrule\hbox{\vrule\squarebox{.667em}\vrule}\hrule}}
\newenvironment{solution}{\begin{trivlist}\item[]{\bf Solution:}}
                      {\qed \end{trivlist}}
\newenvironment{solsketch}{\begin{trivlist}\item[]{\bf Solution Sketch:}}
                      {\qed \end{trivlist}}
\newenvironment{code}{\begin{tabbing}
12345\=12345\=12345\=12345\=12345\=12345\=12345\=12345\= \kill }
{\end{tabbing}}

\newcommand{\eqref}[1]{Equation~(\ref{eq:#1})}

\newcommand{\hint}[1]{({\bf Hint}: {#1})}
%Put more macros here, as needed.
\newcommand{\room}{\medskip\ni}
\newcommand{\brak}[1]{\langle #1 \rangle}
\newcommand{\bit}[1]{\{0,1\}^{#1}}
\newcommand{\zo}{\{0,1\}}
\newcommand{\C}{{\cal C}}

\newcommand{\nin}{\not\in}
\newcommand{\set}[1]{\{#1\}}
\renewcommand{\ni}{\noindent}
\renewcommand{\gets}{\leftarrow}
\renewcommand{\to}{\rightarrow}
\newcommand{\assign}{:=}

\newcommand{\AND}{\wedge}
\newcommand{\OR}{\vee}

\newcommand{\For}{\mbox{\bf For }}
\newcommand{\To}{\mbox{\bf to }}
\newcommand{\Do}{\mbox{\bf Do }}
\newcommand{\If}{\mbox{\bf If }}
\newcommand{\Then}{\mbox{\bf Then }}
\newcommand{\Else}{\mbox{\bf Else }}
\newcommand{\While}{\mbox{\bf While }}
\newcommand{\Repeat}{\mbox{\bf Repeat }}
\newcommand{\Until}{\mbox{\bf Until }}
\newcommand{\Return}{\mbox{\bf Return }}
\newcommand{\Swap}{\mbox{\bf Swap }}

\begin{document}

\ifnum\me=0
%\handout{PS\num}{\today}{Name: Jason Yao}{Due:
%\due}{Solutions to Problem Set \num}
%
%I collaborated with *********** INSERT COLLABORATORS HERE (INDICATING
%SPECIFIC PROBLEMS) *************.
\fi
\ifnum\me=1
\handout{PS\num}{\today}{Name: Jason Yao}{Due: \due}{Solution
{\em Sketches} to Problem Set \num}
\fi
\ifnum\me=2
\handout{PS\num}{\today}{Lecturer: Yevgeniy Dodis}{Due: \due}{Problem
Set \num}
\fi


\newproblem{Frugal Tourist}{12 Points}

You want to travel on a straight line from from city $A$ to city $B$
which is $N$ miles away from $A$. For concreteness, imagine a line
with $A$ being at $0$ and $B$ being at $N$. Each day you can travel at
most $d$ miles (where $0<d<N$), after which you need to stay at an
expensive hotel. There are $n$ such hotels between $0$ and $N$,
located at points $0 <a_1<a_2<\ldots < a_n = N$ (the last hotel is in
$B$). Luckily, you know that $|a_{i+1}-a_i|\le d$ for any $i$ (with
$a_0=0$), so that you can at least travel to the next hotel in one
day. You goal is to complete your travel in the smallest number of
days (so that you do not pay a fortune for the hotels).

Consider the following greedy algorithm: ``Each day, starting at the
current hotel $a_i$, travel to the furthest hotel $a_j$
s.t. $|a_j-a_i|\le d$, until eventually $a_n=N$ is reached''. I.e., if
several hotels are within reach in one day from your current position,
go to the one closest to your destination.

\begin{itemize}
\item[(a)] (6 points) Formally argue that this algorithm is correct using the
``Greedy Stays Ahead'' method.\\
\hint{Think how to define $F_i(Z)$.
For this problem, the name of the method is really appropriate.}

\ifnum\me<2
\begin{solution}
We define $F_i(z) = $distance from B

$F_i(z) = N - a_i$

Show for each step $i$, $greedy_{ai}$ will be further out or equal to $OPT_{ai}$

Base case:
By the definition of the greedy algorithm, we see that $greedy_{a1} \geq OPT_{a1}$

Induction step:
Assuming $greedy_{a1}$ is already ahead from $a_1$, and we continue going d to the next step, $greedy_{ai} \geq OPT_{ai}$.

Since the distance traveled is $\leq d$, $greedy_{a(i + 1)} \geq OPT_{a(i + 1)}$
\end{solution}
\fi

\item[(b)] (6 points) Formally argue that this algorithm is correct using the
``Local Swap'' method. More concretely, given some hypothetical
optimal solution $Z$ of size $k$ and the solution $Z^*$ output by
greedy, define some solution $Z_1$ with the following two properties:
(1) $Z_1$ is no worse than $Z$; (2) $Z_1$ agrees with greedy in the
first day travel plan. After $Z_1$ is defined, define $Z_2$ s.t.: (1)
$Z_2$ is no worse than $Z_1$; (2) $Z_2$ agrees with greedy in the
first two days travel plan. And so on until you eventually reach
greedy.

\ifnum\me<2
\begin{solution}

Z = OPT, Z* = Greedy

$Z_1 = $no worse than Z, so 

$Z_1 = Z^*_1 + Z_2 + ... + Z_k$

$Z_2 = Z^*_1 + Z^*_2 + Z_3 + ... + Z_k$

$Z_p = Z^*_1 + Z^*_2 + ... + Z^*_p + Z_{p + 1} + ... + Z_k$, for p: $1 \leq p \leq k$.

Thus, when p == k,
$Z_k = Z^*_1 + Z^*_2 + ... + Z^*_k$

\end{solution}
\fi

\end{itemize}

\newproblem{Fibonacci meets Huffman} {10 points}

Recall, Fibonacci numbers are defined by $f_0= f_1=1$ and $f_i =
f_{i-1} + f_{i-2}$ for $i\ge 2$.

\begin{itemize}
\item[(a)] (2 points) What is the optimal Huffman code for the following set
of frequencies which are the first $8$ Fibonacci numbers.
\ifnum\me<2
\begin{solution}
\\
\\
\\
\\
\\
\\
\\
\\
\\
\\
\\
\\
\\
\\
\\
\\
\\
\\
\\

\end{solution}
\fi

\item[(b)] (4 points) Let $S_1 = 2  = f_0 + f_1$ and $S_i = S_{i-1} + f_i =
\ldots = f_i + f_{i-1} + \ldots f_1 + f_0$ (for $i>1$) be the sum of the
first $i$ Fibonacci numbers. Prove that $S_i = f_{i+2} -1$ for any $i\ge
1$.

\ifnum\me<2
\begin{solution}

Show $S_i = S_{i - 1} + f_i = f_i - + f_{i - 1} + ... + f_1 + f_0 = f_{i + 2} - 1$

Base case: Let i = 1

$S_1 = f_1 + f_0 = f_3 - 1$

$S_1 = 1 + 1 = 3 - 1$

$S_1 = 2 = 2$

Assume for all $S_i$ is true

$S_{i + 1} = (S_i + f_{i + 1})$

$S_{i + 1} = (f_{i + 2} - 1) + f_{i + 1}$

$S_{i + 1} = f_{i + 3} - 1$

\end{solution}
\fi

\item[(c)] (4 points) Generalize your solution to part (a) to find the shape of
the optimal Huffman code for the first $n$ Fibonacci numbers.
Formally argue that your tree structure is correct, by using part (b).

\ifnum\me<2
\begin{solution}

By part b, we have shown that for all $S_i$, in which $i \geq 2$, $S_i$ has two children, namely $S_{i - 1}$ and $f_i$. Thus the tree structure is that of a very long, unbalanced tree that slopes in one direction. Assuming that this is a BST, it would slop down to to the left, since $f_i$ would be on the right, and $S_{i - 1}$ the child on the left.
\end{solution}
\fi

\end{itemize}

\newproblem{Difficult packaging} {12 points}

You have $n$ items of weights $w_1, \ldots, w_n$ that you want to pack. You can use boxes of capacity $u$ each. (You can assume that $w_i \leq u$ for all $i$). You are allowed to put {\em at most two} items in a box (even if more than two could fit!). Your goal is to pack all $n$ items using the {\em smallest possible} total number of boxes. Consider the following idea for a greedy solution:

(1) Take the lightest remaining item $a$; (2) find the heaviest item $b$ (if it exists) that will fit in one box with $a$; (3) Pack these two elements $a$ and $b$ into one box (if no $b$ exists, just pack $a$); (4) repeat steps (1)-(3) with the remaining items.

\begin{itemize}
\item[(a)] (6 points)
Assume that the input weights are already sorted: $w_1\le w_2\le \ldots \le w_n$. Write the fastest pseudocode you an find to implement the above described solution. (You will get full credit only for an $O(n)$ algorithm.)
\ifnum\me<2
\begin{solution}

\lstinputlisting[language=C]{problem3a.c}

\end{solution}
\fi

\item[(b)] (6 points)
Prove the correctness of this algorithm using Local Swap technique.
\\ \hint{Think of the box containing the lightest item $a$ in the optimum solution.}

\ifnum\me<2
\begin{solution}

For the general case:

Let greedy algorithm be a + b, let OPT algorithm be x + y

Show that all swaps of local hybrids are valid:

case 1:

(a, b), (x, y) $\rightarrow$ (x, b), (a, y) a swaps with x

(x, b), (a, y) $\rightarrow$ (a, b), (x, y) b swaps with y

Show that (x, b) and (a, y) are valid hybrids:

$a + b \leq u \geq x + y$

$a + b + x + y \leq 2u$

a is the lightest item, and b is the heaviest item that still fits.

Thus, $y \leq b$.

$a + b + x + b \leq 2u$

$x + b \leq u$

If $x + b \leq u$, then from $a + b + x + b \leq 2u$,

$a + y \leq u$

Therefore (x, b) and (a, y) are both valid hybrids.

case 2:

(a, b), (x, y) $\rightarrow$ (y, b), (x, a) a swaps with y

(y, b), (x, a) $\rightarrow$ (y, x), (b, a) b swaps with x

Show that (y, b) and (x, a) are valid hybrids:
$a + b \leq u \geq x + y$

$a + b + x + y \leq 2u$

By symmetry of the proof for case 1, and exchanging the symbols, we show that (y, b) and (x, a) are valid hybrids.
\end{solution}
\fi

\item[(c)] ({\bf Extra credit}: ? points)
Help the authors of this problem and come up with GASA proof for this algorithm.

\end{itemize}

\newproblem{Optimal Prefix-Free Encoding} {10 points}

Recall the following two PFEs $E_1$ and $E_2$ discussed in class. Given a message $m$ of length $n$ (known to encoder), $E_1(m) = 0^n ~1~ m$; $E_2(m) = 0^{\lceil \log_2 n \rceil} 1~ \langle n \rangle_2 ~m$, where 
$\langle n \rangle_2$ is the binary representation of integer $n$. Namely, $E_1$ encodes in unary the length of the string and then  the actual string, while the better encoding $E_2$ starts with encoding the length $n$ using $E_1$, and then putting actual string $m$. Recall also the overhead of $E_1$ is $n+1$ and the overhead of $E_2$ is $2\lceil \log_2 n \rceil +1 = 2\log_2 n + O(1)$.

\begin{itemize}
\item[(a)] (5 points)
Extend the above idea one more iteration to design an encoding $E_3$ having overhead of $\log_2 n + 2 \log_2 \log_2 n + O(1)$.
\ifnum\me<2
\begin{solution}

$E_3(m) = 0^{\log_2 \log_2 n}1 <\log_2 n>_2 <n>_2 m$

Let H = overhead,

$H = \log_2n + 2\log_2 \log_2 n + O(1)$
\end{solution}
\fi

\item[(b)] (5 points)
Now write a fully recursive procedure (instead of stopping it after a constant number of iterations) to get an encoding $E^*$ having overhead\footnote{Turns out this overhead is the best possible one can achieve in any PFE.}
$$\log_2 n +  \log_2 \log_2 n + \log_2 \log_2 \log_2 n + \ldots + O(1)$$
where the number of terms in the sum is the number of times one can take $\log_2$ of $n$ until one reaches $1$ (called inverse Ackerman function).

\ifnum\me<2
\begin{solution}

$E_k(m) = 0^{\log_k ... \log n}1<\log_k...\log_1 n>_2
<\log_{k-1} ... \log_1 n>_2 ... <\log n>_2 <n>_2 m $

Let H = overhead,

$H = \log_2n + \log_2\log_2n + ... + O(1)$
\end{solution}
\fi

\end{itemize}

\end{document}
