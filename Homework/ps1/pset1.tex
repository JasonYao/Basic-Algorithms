%me=0 student solutions, me=1 - my solutions, me=2 - assignment
\def\me{0}
\def\num{1}  %homework number
\def\due{Wednesday, February 4}  %due date
\def\course{CSCI-UA.0310-004/005 Basic Algorithms} %course name
\def\name{Jason Yao}
%
\iffalse
INSTRUCTIONS: replace # by the homework number.
(if this is not ps#.tex, use the right file name)

  Clip out the ********* INSERT HERE ********* bits below and insert
appropriate TeX code.  Once you are done with your file, run

  ``latex ps#.tex''

from a UNIX prompt.  If your LaTeX code is clean, the latex will exit
back to a prompt.  To see intermediate results, type

  ``xdvi ps#.dvi'' (from UNIX prompt)
  ``yap ps#.dvi'' (if using MikTex in Windows)

after compilation. Once you are done, run

  ``dvips ps#.dvi''

which should print your file to the nearest printer.  There will be
residual files called ps#.log, ps#.aux, and ps#.dvi.  All these can be
deleted, but do not delete ps1.tex. To generate postscript file ps#.ps,
run

  ``dvips -o ps#.ps ps#.dvi''

I assume you know how to print .ps files (``lpr -Pprinter ps#.ps'')
\fi
%
\documentclass[11pt]{article}
\usepackage{amsfonts}
\usepackage{latexsym}
\usepackage{listings}
\setlength{\oddsidemargin}{.0in}
\setlength{\evensidemargin}{.0in}
\setlength{\textwidth}{6.5in}
\setlength{\topmargin}{-0.4in}
\setlength{\textheight}{8.5in}

\newcommand{\handout}[5]{
   \renewcommand{\thepage}{#1, Page \arabic{page}}
   \noindent
   \begin{center}
   \framebox{
      \vbox{
    \hbox to 5.78in { {\bf \course} \hfill #2 }
       \vspace{4mm}
       \hbox to 5.78in { {\Large \hfill #5  \hfill} }
       \vspace{2mm}
       \hbox to 5.78in { {\it #3 \hfill #4} }
      }
   }
   \end{center}
   \vspace*{4mm}
}

\newcounter{pppp}
\newcommand{\prob}{\arabic{pppp}}  %problem number
\newcommand{\increase}{\addtocounter{pppp}{1}}  %problem number

%first argument desription, second number of points
\newcommand{\newproblem}[2]{
\ifnum\me=0
\ifnum\prob>0 \newpage \fi
\increase
\setcounter{page}{1}
\handout{\name, Homework \num, Problem \arabic{pppp}}{\today}{Name: \name}{Due:
\due}{Solutions to Problem \prob\ of Homework \num\ (#2)}
\else
\increase
\section*{Problem \num-\prob~(#1) \hfill {#2}}
\fi
}

%\newcommand{\newproblem}[2]{\increase
%\section*{Problem \num-\prob~(#1) \hfill {#2}}
%}

\def\squarebox#1{\hbox to #1{\hfill\vbox to #1{\vfill}}}
\def\qed{\hspace*{\fill}
        \vbox{\hrule\hbox{\vrule\squarebox{.667em}\vrule}\hrule}}
\newenvironment{solution}{\begin{trivlist}\item[]{\bf Solution:}}
                      {\qed \end{trivlist}}
\newenvironment{solsketch}{\begin{trivlist}\item[]{\bf Solution Sketch:}}
                      {\qed \end{trivlist}}
\newenvironment{code}{\begin{tabbing}
12345\=12345\=12345\=12345\=12345\=12345\=12345\=12345\= \kill }
{\end{tabbing}}

\newcommand{\eqref}[1]{Equation~(\ref{eq:#1})}

\newcommand{\hint}[1]{({\bf Hint}: {#1})}
%Put more macros here, as needed.
\newcommand{\room}{\medskip\ni}
\newcommand{\brak}[1]{\langle #1 \rangle}
\newcommand{\bit}[1]{\{0,1\}^{#1}}
\newcommand{\zo}{\{0,1\}}
\newcommand{\C}{{\cal C}}

\newcommand{\nin}{\not\in}
\newcommand{\set}[1]{\{#1\}}
\renewcommand{\ni}{\noindent}
\renewcommand{\gets}{\leftarrow}
\renewcommand{\to}{\rightarrow}
\newcommand{\assign}{:=}

\newcommand{\AND}{\wedge}
\newcommand{\OR}{\vee}

\newcommand{\For}{\mbox{\bf For }}
\newcommand{\To}{\mbox{\bf to }}
\newcommand{\Do}{\mbox{\bf Do }}
\newcommand{\If}{\mbox{\bf If }}
\newcommand{\Then}{\mbox{\bf Then }}
\newcommand{\Else}{\mbox{\bf Else }}
\newcommand{\While}{\mbox{\bf While }}
\newcommand{\Repeat}{\mbox{\bf Repeat }}
\newcommand{\Until}{\mbox{\bf Until }}
\newcommand{\Return}{\mbox{\bf Return }}
\newcommand{\Swap}{\mbox{\bf Swap }}

\begin{document}

\ifnum\me=0
%\handout{PS\num}{\today}{Name: Jason Yao}{Due:
%\due}{Solutions to Problem Set \num}
%
%I collaborated with *********** INSERT COLLABORATORS HERE (INDICATING
%SPECIFIC PROBLEMS) *************.
\fi
\ifnum\me=1
\handout{PS\num}{\today}{Name: Yevgeniy Dodis}{Due: \due}{Solution
{\em Sketches} to Problem Set \num}
\fi
\ifnum\me=2
\handout{PS\num}{\today}{Lecturer: Yevgeniy Dodis}{Due: \due}{Problem
Set \num}
\fi

\newproblem{Adding Binary Integers}{5 points}

You are given two binary integers $a$ and $b$, where $a$ has $n$ bits, and $b$ has $m$ bits,
where $n\ge m$. These integers are stored in two arrays $A[1\ldots n]$
and $B[1\ldots m]$, {\em in reverse order}. For example, if $a=111000$
and $b=1110$, then $A[1]=A[2]=A[3]=0$, $A[4]=A[5]=A[6]=1$, $B[1]=0$,
$B[2]=B[3]=B[4]=1$. Your goal is to produce an array $C[1\ldots n+1]$
which stores the sum $c$ of $a$ and $b$. For example, $111000+1110 =
1000110$, meaning that $C[1\ldots 7] = 0110001$.

Write the pseudocode to produce $C$. Why was it convenient to store
the arrays ``backwards''?

\lstset{ %
breaklines=true,        % sets automatic line breaking
breakatwhitespace=false,    % sets if automatic breaks should only happen at whitespace
showtabs=false,                 % show tabs within strings adding particular underscores
frame=single,           % adds a frame around the code
tabsize=2,          % sets default tabsize to 2 spaces
}


\ifnum\me=0
\begin{solution}


\begin{lstlisting}
binAdd(a,b)

n = a.length
m = b.length
maxLength = abs(n - m)
A = new [n]
B = new [m]
C = new [maxLength + 1]

For i = n - 1 to 0
	A[i] = a % 10
	a /= 10
endFor

For j = m - 1 to 0
	B[j] = b % 10
	b /= 10
endFor

index = 0
carry = 0

while index <= n && index <= m
	if A[index] + B[index] == 2
		if carry == 1
			C[index] = 1
		endif
		if carry = 0
			C[index] = 1 
		endif
		++index
	endif
	
	if A[index] + B[index] == 1
		if carry == 1
			C[index] = 1
			carry = 0
		endif
		if carry == 0
			C[index] == 1
		endif
		++index
	endif
	
	if A[index] + B[index] == 0
		if carry == 1
			C[index] = 1
			carry = 0
		endif
		if carry == 0
			C[index] = 0
		endif
		++index
	endif
endwhile

while index <= n
	if carry == 1
		if A[index] == 0
			C[index] = 1
			carry = 0
		endif
		if A[index] == 1
			C[index] = 1
		endif
		++index
	endif
	if carry == 0
		C[index] = A[index]
		++index
	endif
endwhile

while index <= m
	if carry == 1
		if B[index] == 0
			C[index] = 1
			carry = 0
		endif
		if B[index] == 1
			C[index] = 1
		endif
		++index
	endif
	if carry == 0
		C[index] = A[index]
		++index
	endif
endwhile

if carry == 1
	C[maxLength] = 1
endif
\end{lstlisting}

It is convenient to store the arrays 'backwards' because we start binary addition from the last integer, allowing us to progress through the array normally instead of going from the other end.
\end{solution}
\fi

\newproblem{Insertion Sort Variation}{5 points}

In class we learned how to implement insertion sort by comparing the
key element to the largest element in the sorted portion of the array,
and moving that element to the right, if it was larger than the key,
and then comparing the key to successively smaller elements until the
right position is found.

Implement a variation of insertion sort, in which you instead compare
the key to the smallest element in the sorted portion of the array and
then iterate by comparing to successively larger elements.
How does this algorithm compare in terms of efficiency to the
traditional insertion sort?

\ifnum\me<2
\begin{solution}


\begin{lstlisting}
IS(A[])
for i = 1 to A.length - 1
	key = A[i]
	j = 0
	while j < i && A[j + 1] < key
		A[j] = A[j + 1]
		++j
	endwhile
	A[j] = key
endfor
\end{lstlisting}
The algorithm is of comparable efficiency to that of normal insertion sort, due to the fact that the insertion sort is simply starting the comparison from the other end, while still having to go through O($n^2$)
\end{solution}
\fi

\newproblem{Asymptotic Comparisons}{10 Points}

For each of the following pairs of functions $f(n)$ and $g(n)$, state
whether $f$ is $O(g)$; whether $f$ is $o(g)$; whether $f$ is
$\Theta(g)$; whether $f$ is $\Omega(g)$; and whether $f$ is
$\omega(g)$.  (More than one of these can be true for a single pair!)

\begin{itemize}
\item[(a)] $f(n) = 13 n^{19} + 92$; $g(n) = \frac{n^{24}-n^{23}+5}{5n^4+2000}$.

\ifnum\me<2
\begin{solution}
f(n) $\approx n^{19}$, g(n) $\approx n^{20}$\\
$\lim_{n\to\infty} \frac{f(n)}{g(n)}$,
$\lim_{n\to\infty} \frac{n^{19}}{n^{20}}$,
$\lim_{n\to\infty} \frac{1}{n}$,
$\lim_{n\to\infty} 0$\\

Thus f(n) = o(g(n)), and by definition, f(n) = O(g(n))
\end{solution}
\fi

\item[(b)]  $f(n) = \log(n^{19}+3n)$; $g(n) = \log(n^{0.2}-1)$.


\ifnum\me<2
\begin{solution}
f(n) $\approx log(n^{19})$, g(n) $\approx log(n^{0.2})$\\
$\lim_{n\to\infty} \frac{f(n)}{g(n)}$,
$\lim_{n\to\infty} \frac{log(n^{19})}{log(n^{0.2})}$,
$\lim_{n\to\infty} \frac{19*log(n)}{0.2*log(n)}$,
$\lim_{n\to\infty} \frac{19}{0.2}$,
$\lim_{n\to\infty} 95$ \\

Thus f(n) = $\theta$(g(n))
\end{solution}
\fi

\item[(c)]  $f(n) = \log (5^{n}+n)$; $g(n) = \log(n^{20})$.

\ifnum\me<2
\begin{solution}
f(n) $\approx log(5^{n})$, g(n) $= log(n^{20})$\\
$\lim_{n\to\infty} \frac{f(n)}{g(n)}$,
$\lim_{n\to\infty} \frac{log(5^n)}{log(n^{20})}$,
$\lim_{n\to\infty} \frac{n*log(5)}{20*log(n)}$,
$\lim_{n\to\infty} \infty$,\\

Thus f(n) = $\omega(g(n))$, and by definition, f(n) = $\Omega(g(n))$
\end{solution}
\fi

\item[(d)] $f(n) = n^{4} \cdot 3^{n}$; $g(n) = n^{3} \cdot 4^{n}$.

\ifnum\me<2
\begin{solution}
$\lim_{n\to\infty} \frac{f(n)}{g(n)}$,
$\lim_{n\to\infty} \frac{3^{n} \cdot n^{4}}{4^{n} \cdot n^{3}}$,
$\lim_{n\to\infty} (\frac{3}{4})^{n}$
$\lim_{n\to\infty} 0$\\

Thus f(n) = o(g(n)), and by definition, f(n) = O(g(n))
\end{solution}
\fi

\item[(e)] $f(n) = (n^{n})^{2}$; $g(n)=n^{(n^{2})}$.

\ifnum\me<2
\begin{solution}
f(n) $= n^{2n}$, g(n) $= n^{n^2}$\\
$\lim_{n\to\infty} \frac{f(n)}{g(n)}$,
$\lim_{n\to\infty} \frac{n^{2n}}{n^{n^2}}$,
$\lim_{n\to\infty} (\frac{2n*log(n)}{n^2 * log(n)})^{n}$
$\lim_{n\to\infty} 0$\\

Thus f(n) = o(g(n)), and by definition, f(n) = O(g(n))
\end{solution}
\fi

\end{itemize}

\newproblem{The Same or Not the Same?}{10 points}

The following two functions both take as arguments two $n$-element
arrays $A$ and $B$:

\begin{code}
{\sc Magic-1}$(A,B,n)$ \\
\>  \For $i = 1$ \To $n$\\
\>\>  \For $j = 1$ \To $n$\\
\>\>\>  \If $A[i] \ge B[j]$ \Return FALSE\\
\>  \Return TRUE\\
\end{code}

\begin{code}
{\sc Magic-2}$(A,B,n)$ \\
\>  $temp := A[1]$\\
\> \For $i = 2$ \To $n$\\
\>\>  \If $A[i] > temp$ \Then $temp := A[i]$\\
\> \For $j = 1$ \To $n$\\
\>\> \If $temp \ge B[j]$ \Return FALSE\\
\>  \Return TRUE\\
\end{code}


\begin{itemize}

\item[(a)] (2 points) It turns out both of these procedures return
  TRUE if and only if the same `special condition' regarding the
    arrays $A$ and $B$ holds. Describe this `special condition' in
  English.

\ifnum\me<2
\begin{solution}
The special condition is when A[i] < B[j]
\end{solution}
\fi

\item[(b)] (5 points) Analyze the worst-case running time for both
  algorithms in the $\Theta$-notation. Which algorithm would you
  chose? Is it the one with the shortest code (number of lines)?

\ifnum\me<2
\begin{solution}
As Magic-1 has a worst-case running time of $\Theta(n^2)$, and Magic-2 has a worst-case running time of $\Theta(n)$, that means that Magic-2 is much more efficient than Magic-1, and thus I would choose Magic-2 as the optimal solution.
\end{solution}
\fi

\item[(c)] (3 points) Does the situation change if we consider the
  best-case running time for both algorithms?

\ifnum\me<2
\begin{solution}
No, the situation does not change if we consider the best-case running time for the algorithms, since both running times would remain the same as their worst-case scenario counterparts.
\end{solution}
\fi

\end{itemize}

\end{document}
